\documentclass{beamer}
\usepackage{listings}
\lstset{
%language=C,
frame=single, 
breaklines=true,
columns=fullflexible
}
\usepackage{subcaption}
\usepackage{setspace}
\usepackage{url}
\usepackage{tikz}
\usepackage{tkz-euclide} % loads  TikZ and tkz-base
%\usetkzobj{all}
\usepackage[utf8]{inputenc}
\usepackage{longtable}
\usetikzlibrary{calc,math}
\usepackage{float}
\usetheme{Berlin}
\usepackage{graphicx}
\usepackage{hyperref}
\usecolortheme{beaver}


\newcommand\norm[1]{\left\lVert#1\right\rVert}
\renewcommand{\vec}[1]{\mathbf{#1}}
\usepackage[export]{adjustbox}
\usepackage[utf8]{inputenc}
\usepackage{amsmath}
%\usetheme{Boadilla}
\newcommand\mytextbullet{\leavevmode%
\usebeamertemplate{itemize item}\hspace{.5em}}

\bibliographystyle{IEEEtran}

\usepackage{color}

\title{7-Segment Display using Flysky Transmitter}
\author{B603 Lab}
%{\and} \\
%\vspace{10pt}
%{Supervisor:} \\
%{Dr. GVV Sharma} }

\institute{Indian Institute of Technology, Hyderabad.}
\date{\today}

\begin{document}


\begin{frame}
\titlepage
\end{frame}


\section{Components}
\begin{frame}
\frametitle{Components}
\begin{columns}
\column{1\textwidth}
  \begin{enumerate}
  \item Flysky Receiver and Transmitter
  \vspace{10pt}
  \item ESP32
  \vspace{10pt}
  \item 7 Segment Display
  
  
  \end{enumerate}
  
\end{columns}

\end{frame}

\section{Pinout Diagram of 7-Segment Display}
\begin{frame}
\frametitle{Pinout Diagram of 7-Segment Display}
\begin{columns}
\column{1\textwidth}
\begin{figure}[h!]
  \centering
  \begin{subfigure}[b]{0.75\linewidth}
    \includegraphics[width=\linewidth]{./figs/7-Segment-CA-Pinout-2.jpg}
%    \caption{Coffee.}
  \end{subfigure}

 % \caption{Flysky Receiver}
%  \label{fig:axis}
\end{figure}



%\caption{Flysky Receiver}
  
\end{columns}
%\vspace{25px}



\end{frame}

\section{Wiring Diagram: Flysky-Rx and ESP32}
\begin{frame}
\frametitle{Wiring Diagram: Flysky-Rx and ESP32}
\begin{columns}
\column{0.4\textwidth}
  
  \begin{tabular} { | c  | c  |  }
%\hline
%\multicolumn{3} { | c | }{Books}\\
\hline
\textbf{Flysky-Rx pin} & \textbf{ESP32 pin} \\
\hline
Channel 2 & Pin-14  \\
\hline
Channel 4 & Pin-15\\
\hline
GND & GND  \\
\hline
Vin & 5v\\
\hline
\end{tabular}


  
 
  
\end{columns}



\end{frame}

\section{Wiring Diagram: 7-Segment Display and ESP32}
\begin{frame}
\frametitle{Wiring Diagram: 7-Segment Display and ESP32}
\begin{columns}
\column{0.4\textwidth}
  
  \begin{tabular} { | c  | c  |  }
%\hline
%\multicolumn{3} { | c | }{Books}\\
\hline
\textbf{7-Segment pin} & \textbf{ESP32 pin} \\
\hline
Pin 1 & P-19\\
\hline
Pin 2 & P-17\\
\hline
Pin 4 & P-18\\
\hline
Pin 6 & P-18\\
\hline
Pin 7 & P-16  \\
\hline
Pin 8 & 3.3V\\
\hline
Pin 9 & P-19\\
\hline

\end{tabular}


  
 
  
\end{columns}



\end{frame}


\section{Wiring Diagram: Flysky-Rx and Arduino}
\begin{frame}
\frametitle{Wiring Diagram: Flysky-Rx and Arduino}
\begin{columns}
\column{0.4\textwidth}
  
  \begin{tabular} { | c  | c  |  }
%\hline
%\multicolumn{3} { | c | }{Books}\\
\hline
\textbf{Flysky-Rx pin} & \textbf{Arduino pin} \\
\hline
Channel 2 & Pin-10  \\
\hline
Channel 4 & Pin-11\\
\hline
GND & GND  \\
\hline
Vin & 5v\\
\hline
\end{tabular}
  
\end{columns}

\end{frame}


\section{Wiring Diagram: 7-Segment Display and Arduino}
\begin{frame}
\frametitle{Wiring Diagram: 7-Segment Display and Arduino}
\begin{columns}
\column{0.4\textwidth}
  
  \begin{tabular} { | c  | c  |  }
%\hline
%\multicolumn{3} { | c | }{Books}\\
\hline
\textbf{7-Segment pin} & \textbf{Arduino pin} \\
\hline
Pin 1 & P-2\\
\hline
Pin 2 & P-3\\
\hline
Pin 4 & P-4\\
\hline
Pin 6 & P-4\\
\hline
Pin 7 & P-5  \\
\hline
Pin 8 & 3.3V\\
\hline
Pin 9 & P-2\\
\hline

\end{tabular}


  
 
  
\end{columns}



\end{frame}

\section{Binding Tx. and Rx.}
\begin{frame}
\frametitle{Binding Tx. and Rx.}
\begin{columns}
\column{1\textwidth}

  \begin{itemize}
  \item  Check the reference number of the receiver and transmitter. If they are same, then no need of binding.
  \vspace{10pt}
  \item If the reference numbers are not same then binding of Rx and Tx is needed.
  
  
  \end{itemize}
\end{columns}
%\vspace{25px}



\end{frame}

\section{Binding Tx. and Rx.}
\begin{frame}
\frametitle{Binding Tx. and Rx.}
\begin{columns}
\column{0.4\textwidth}
\begin{figure}[h!]
  \centering
  \begin{subfigure}[b]{0.75\linewidth}
    \includegraphics[width=\linewidth]{./figs/rx1.png}
%    \caption{Coffee.}
  \end{subfigure}

  \caption{Flysky Receiver}
%  \label{fig:axis}
\end{figure}

\column{0.4\textwidth}
\begin{figure}[h!]
  \centering
  \begin{subfigure}[b]{0.75\linewidth}
    \includegraphics[width=\linewidth]{./figs/rx1_updated.png}
%    \caption{Coffee.}
  \end{subfigure}

  \caption{Flysky Receiver}
%  \label{fig:axis}
\end{figure}
%\caption{Flysky Receiver}
  
\end{columns}
%\vspace{25px}



\end{frame}

\section{Binding Tx. and Rx.}
\begin{frame}
\frametitle{Binding Tx. and Rx.}
\begin{columns}
\column{1\textwidth}

  \begin{itemize}
  \item  Short the GND and CHANNEL pin of B/VCC of the receiver.
  \vspace{10pt}
  \item Connect Vin pin with 5V pin of ESP32 and GND pin to GND pin of ESP32.
  \vspace{10pt}
  \item Press the bind key of the transmitter.
  
  
  \end{itemize}
\end{columns}



\end{frame}

\section{Code}
\begin{frame}
\frametitle{Code}
\begin{columns}
\column{1\textwidth}

  \begin{itemize}
  \item  Flash the code from arduino IDE to ESP32.\\
  \url{https://github.com/JayatiD93/7SegmentDisplay_ESP32/blob/main/flysky_7Segment.ino}
  
  
  \end{itemize}
  \  
\end{columns}
%\vspace{25px}



\end{frame}
































\end{document}

